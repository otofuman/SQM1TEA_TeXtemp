%%%%%%%%%%文書の構成、使用パッケージ宣言
\documentclass[
  11pt,
  a4paper,
  ]{jsreport}
\usepackage{SQM1TEAcom191208}

%%%%%%%%%%タイトルと著者:表紙に表示される
\title{タイトル}
\author{所属 学生番号 氏名}

%%%%%%%%%%文書
%%%%% \input{SQM1TEA[大問番号]_[小問番号]}は各大問の各小問の
%%%%% 解答ファイル「SQM1TEA[大問番号]_[小問番号].tex」の取り込みを命令します。
%%%%% 別途{subsection}環境内に解答を記述した上記名のファイルを用意してください。
%%%%% また各\input{}の行頭の%は存在しないファイルを読まないためのものです。
%%%%% ファイルを作成したら、行頭の%を削除してください。
%%%%%%%%%%

\begin{document}
  \maketitle%%%%%%%%%%表紙
  \clearpage%%%%%%%%%%改ページ
  \begin{section}{}%%%%%%%%%%大問[1]
    %\input{Pr1/1_1}
    %\input{Pr1/1_2}
    %\input{Pr1/1_3}
    %\input{Pr1/1_4}
    %\input{Pr1/1_5}
    %\input{Pr1/1_6}
  \end{section}
  \setcounter{subsection}{0}%%%%%%%%%%小節カウンタのリセット
  \begin{section}{}%%%%%%%%%%大問[2]
    %\input{Pr2/2_1}
    %\input{Pr2/2_2}
    %\input{Pr2/2_3}
    %\input{Pr2/2_4}
    %\input{Pr2/2_5}
    %\input{Pr2/2_6}
    %\input{Pr2/2_7}
  \end{section}
  \setcounter{subsection}{0}%%%%%%%%%%小節カウンタのリセット
  \begin{section}{}%%%%%%%%%%大問[3]
    %\input{Pr3/3_1}
    %\input{Pr3/3_2}
    %\input{Pr3/3_3}
    %\input{Pr3/3_4}
    %\input{Pr3/3_5}
  \end{section}
  \setcounter{subsection}{0}%%%%%%%%%%小節カウンタのリセット
  \begin{section}{}%%%%%%%%%%大問[4]
    %\input{Pr4/4_1}
    %\input{Pr4/4_2}
    %\input{Pr4/4_3}
    %\input{Pr4/4_4}
    %\input{Pr4/4_5}
    %\input{Pr4/4_6}
  \end{section}
\end{document}
